\section{Canonical transformations}

The action $A_H$ of the Hamiltonian $H(q, p, t)$ is

\begin{equation}
    A_H = \int_{t_1}^{t_2} p \cdot \Dot{q} - H \dd{t}
\end{equation}

We want to change the coordinates, so that we get $K(Q, P, t)$ as the new Kamiltonian. We require $A_H = c A_K + \Delta F$, where $c$ and $\Delta F$ are real constants.

Then

\begin{equation}
    A_H - c A_K = \int_{t_1}^{t_2}
    p \cdot \dd{\Dot{q}} + H - c (P \cdot \dd{\Dot{Q}} + K) \dd{t} 
    = \Delta F
\end{equation}

and looking at the corresponding differential forms:

\begin{align}
    \dd{F} (q, Q, t) &= p \cdot \dd{q} + H - c P \cdot \dd{Q} - cK \dd{t} \\  
    &= p \cdot \dd{q} - c P \cdot \dd{Q} + (H-cK)\dd{t}
\end{align}

The \emph{canonicity} of the transformation corresponds to the \emph{closure} of the form $\dd{F}$.
Some other kinds of transformations:

\begin{align}
    S(q, P, t) &= F(q, Q, t) + c Q \cdot P \\
    \dd{S} &= p \cdot \dd{q} + cQ \cdot \dd{P} + (cK-H) \dd{t}
\end{align}

