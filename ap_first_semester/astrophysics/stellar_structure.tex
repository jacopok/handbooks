\documentclass[main.tex]{subfiles}
\begin{document}

\section{Energy transport in stellar interiors}

This section contains my notes while studying the fifth chapter in the  ``Theoretical Astrophysics'' lecture notes by Paola Marigo. 

Energy transport phenomena in stellar interiors are classified into: 
\begin{enumerate}
  \item those which involve stochastic thermal motion of the particles: this is \emph{heat conduction} if it involves the motion of gas and dust particles, and \emph{radiative diffusion} if it involves the motion of photons;
  \item those which involve bulk motion of particles: this is called \emph{convection}. 
\end{enumerate}

The local form of the first law of thermodynamics can be written in terms of the density: we know that the expression for infinitesimal work is \(\delta W = P \delta V\), but the volume is given by \(V = m / \rho \). So, \(\delta V = \delta (m/\rho ) = - m / \rho^2 \times \delta \rho \). So we can express the First Law as:
%
\begin{align}
\delta U &= \delta Q - \delta W = \delta Q + \frac{Pm}{\rho^2} \delta \rho   \\
\delta u &= \delta q + \frac{P}{\rho^2} \delta \rho 
\,,
\end{align}
%
where we introduced the internal energy per unit mass \(u = U/m\) and the heat per unit mass \(q = Q/m\). 

\end{document}