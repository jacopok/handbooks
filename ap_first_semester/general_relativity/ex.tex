\documentclass[main.tex]{subfiles}
\begin{document}

\section{Exercises}

5.1: 
%
\begin{align}
R_{\mu \nu \rho \sigma } = - 2 g_{[\mu | [\rho|,\nu ] |\sigma ]}
\,,
\end{align}
%
5.3: the plane is \(\dd{s^2} = y^{-2} \dd{s^2}_{E}\), the equations of motion read 
%
\begin{align}
y \ddot{y} = \dot{y}^2 - \dot{x}^2
\qquad \text{and} \qquad
y \ddot{x} - 2 \dot{x} \dot{y}
\,,
\end{align}
%
and the conserved quantity comes the \((1,0)\) KV (\(\dot{x} / y^2\)) plus \(y = \norm{u}\). Then, substitute for circles.

7.3: need the \emph{derivative} of the normalization. Easier to start from 
%
\begin{align}
u = \frac{GM}{l^2} + 3GM u^2
\,,
\end{align}
calculate \(\dv*{\varphi }{\tau }\) from \(l\). 

Use \(\Omega \) to calculate \(\dv*{t}{\tau }\). 

8.2: \(a\) is just \(a^{r}\).

9.1: just need to evaluate the potential at \(R = GM\), it is there that it equals \((e^2 - 1)/2\).


9.2: recall \(r^2+a^2 = 2GMr\); form a square for the null vector. \(l = (1, 0,0,\Omega _{\text{hor}})\). 

10.1 final: \(1+R\) is always a factor.

10.2: watch out for \(T_{ij} = p h_{ij} \neq p \delta_{ij}\). 

\end{document}