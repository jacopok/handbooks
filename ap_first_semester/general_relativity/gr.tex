\documentclass[main.tex]{subfiles}
\begin{document}

\section{Theory}

\paragraph{Length contraction and time dilation}

\begin{align}
\dd{\tau } = \frac{ \dd{t}}{\gamma } 
\qquad \text{and} \qquad
\dd{s} = \gamma \dd{x}
\,.
\end{align}



\paragraph{Energy momentum tensor}

\begin{align}
\Delta p^{\alpha } = T^{\alpha \beta } n_{\beta } \Delta V
\,,
\end{align}
%
for non interacting dust 
%
\begin{align}
T^{\alpha \beta } = n_{*} m u^{\alpha } u^{\beta }
\,.
\end{align}


\paragraph{Variational principle}

The proper time is:
\(\dd{\tau }^2 = - \dd{s^2} = -\eta_{\mu \nu } \dd{x^{\mu}} \dd{x^{\nu }} \), so the variational principle is 
%
\begin{align}
\delta \qty(\int_{A}^{B} \dd{\tau }) = 0 
\iff 
\dv{ u^{\mu}}{\tau} = 0
\,.
\end{align}

In general 
%
\begin{align}
u^{\mu } \nabla_{\mu } u^{\nu } = \dv[2]{x^{\nu }}{\tau } 
+ \Gamma^{\mu }_{\alpha \beta } u^{\alpha} u^{\beta }
\iff 
\delta \qty(\int_{A}^{B} \dd{\tau }) = 0
\,,
\end{align}
%
where \(\dd{\tau } = - g_{\mu \nu } \dd{x^{\mu }} \dd{x^{\nu }}\).

Lagrangian view: 
%
\begin{align}
\mathscr{L} = \dv{\tau }{\sigma }= \sqrt{- g_{\mu \nu } \dv{x^{\mu }}{\sigma } \dv{x^{\nu }}{\sigma }}
\,.
\end{align}

\paragraph{Light}

The 4-velocity is \emph{defined} as 
%
\begin{align}
k^{\mu } = (\omega , \vec{k})
\,,
\end{align}
%
with \(\abs{\vec{k}} = \omega \). 
Doppler effect: the observed frequency is \(-u^{\mu } k_{\mu }\), so observers moving with different velocities see different wavevectors.

\paragraph{Observers}

The measured energy of a particle with \(p^{\mu }\) by an obs with velocity \(u^{\mu }\) is \(E = - u^{\mu } p_{\mu }\). 

\paragraph{Gravitational time dilation}

\begin{align}
\frac{\Delta \tau _A}{ 1 + \Phi_{A}}
=\frac{\Delta \tau _B}{ 1 + \Phi_{B}}
\,,
\end{align}
%
and in the inertial frame \(\Delta t = \Delta \tau \) for both.

\paragraph{Christoffel}

\begin{align}
\Gamma^{\mu }_{\nu \rho } = \frac{1}{2} g^{\mu \alpha }
\qty(g_{\alpha \nu , \rho } + g_{\alpha \rho , \nu } - g_{\nu \rho , \alpha })
\,.
\end{align}

Defined by assuming \(\nabla_{\mu } A_{\nu }\) is a tensor, and the tensor differentiation law: 
%
\begin{align}
\nabla_{\mu } A_{\nu } = 
\partial_{\mu } A_{\nu } 
- \Gamma^{\rho }_{\mu \nu } A_{\rho }
\qquad \text{and} \qquad
\nabla_{\mu } A^{\nu } =
\partial_{\mu } A^{ \nu } 
+ \Gamma^{\nu }_{\mu \rho } A^{\rho }
\,.
\end{align}

\paragraph{Riemann tensor}


%
\begin{align}
[\nabla_{\mu }, \nabla_{\nu } ] V^{\alpha } = R_{\beta \mu \nu }^{\alpha } V^{\beta }
\,,
\end{align}
%
where 
%
\begin{align}
R_{\nu \rho \sigma }^{\mu } = -2 \qty(\Gamma^{\mu }_{\nu [\rho , \sigma ]}  + \Gamma^{\beta }_{\nu [\rho } \Gamma^{\mu }_{\sigma ] \beta })
\,,
\end{align}

\paragraph{Parallel transport}

\begin{align}
u^{\mu } \nabla_{\mu } V^{\nu } =0
\,,
\end{align}

\paragraph{Einstein Field Equations}

%
\begin{align}
R_{\mu \nu } - \frac{1}{2} g_{\mu \nu } R = \frac{T_{\mu \nu } }{M_P^2}
\,,
\end{align}
%
where 
\(M_P = 1 / \sqrt{8 \pi G}\) is the reduced Planck mass.

\paragraph{Schwarzschild}

\(R_{00} =0\) and \(R_{11} =0 \) imply 
%
\begin{align}
(AB)' = 0
\,,
\end{align}
%
while \(R_{22} =0\) implies 
%
\begin{align}
1 = B + r \frac{B'}{B}
\,.
\end{align}

\(R_{33} =0\) is not linearly independent.  

Limits to use: 
\begin{enumerate}
  \item \(g_{00} \rightarrow -1\) and \(g_{11} \rightarrow 1\) as \(r \rightarrow \infty \);
  \item \(g_{00} \sim -(1 + 2 \Phi )\) as \(M \rightarrow 0\).
\end{enumerate}


\begin{align}
\dd{s^2} = - \dd{t^2} \qty(1 - \frac{2GM}{r})
+ \frac{ \dd{r^2}}{1 - \frac{2GM}{r}} + r^2 \dd{\Omega^2}
\,.
\end{align}
%

\paragraph{Classical orbits}

Pose \(L = r v_{\varphi }\), which is conserved. 

\begin{align}
\Omega = \sqrt{\frac{GM}{r^3}}
\,
\end{align}
%
follows from equating the forces.
The effective potential is the Schwarzchild one without the \(r^{-3}\) term.

\paragraph{Schwarzschild orbits}

\(u^{\mu } u_{\mu } = -1\) plus the two Killing vectors: 
%
\begin{align}
\frac{1}{2}\dot{r}^2 + V _{\text{eff}} (r) = \frac{e^2-1}{2}
\,,
\end{align}
%
with
%
\begin{align}
V _{\text{eff}} = - \frac{GM}{r} + \frac{l^2}{2 r^2} \qty(1 - \frac{2GM}{r})
\,,
\end{align}
%
then: change variable to \(u = 1/r\), and
%
\begin{align}
\dv{}{\tau } = \frac{l}{r^2} \dv{}{\varphi }
\,.
\end{align}
%
\begin{align}
\dv[2]{u}{\varphi } + u = \frac{GM}{l^2} + 3GMu^2
\,,
\end{align}
%
where \(u = 1/r\). Upon perturbation, we get 
%
\begin{align}
\delta \varphi = 6 \pi \qty(\frac{GM}{l})^2
\,,
\end{align}
%

\paragraph{Radial infall}

\(e = 1\), \(l =0 \). The horizon is regular. 
We recover the escape velocity: An observer sees 
%
\begin{align}
E = - u^{\mu } _{\text{obs}} p_{\mu } = \frac{m}{\sqrt{1 - \frac{2GM}{r}}} \overset{!}{=} m \gamma 
\,,
\end{align}
%
which means \(v = \sqrt{2GM/r}\).

\paragraph{Impact parameter}

\(u^2= 0\) (light), impact parameter \(b = l/e\): to prove it draw, recall \(l = \dv*{\varphi }{\lambda }\) and \(e =
\dv*{t}{\lambda }\). 
The light equation is 
%
\begin{align}
\frac{1}{l^2} \dot{r}^2  + W _{\text{eff}} (r)
= \frac{1}{b^2}
\,,
\end{align}
%
where 
%
\begin{align}
W _{\text{eff}} = \frac{1}{r^2} \qty(1 - \frac{2GM}{r})
\,.
\end{align}

%
\begin{align}
\dv[2]{u}{\varphi } + u = 3GM u^2
\,,
\end{align}
%
perturb \(u_0 = b^{-1} \sin(\varphi )\) with \(w/b\). 
Ansatz \(w = A + B \sin^2\varphi \). Find \(u(\varphi )\), 
pose \(u=0\), discard \(\sin^2\) and linearize \(\sin\). 

\paragraph{Geodetic precession}

\(s \cdot u =0\), \(u^2 = -1 \) and \(u\) is equatorial circular orbit
%
\begin{align}
u^{\nu } \nabla_{\nu } s^{\mu } = 0
\,,
\end{align}
%
gives 
%
\begin{align}
\overline{\Omega} = \Omega \sqrt{1 - \frac{3GM}{r}}
\,,
\end{align}
%
so, compute \(\cos(\Delta \varphi )\) as ratio of radial components and pose \(\Delta \varphi = - \overline{\Omega} t \). 

\paragraph{Slowly rotating metric --- Lense-Thirring precession}

Work in orders of negative powers of \(c\), so no \(M\) and Minkowski plus \(g_{03} \) terms, \(g_{03} = -2 GJ \sin^2\theta /r\).

Turn \(\dd{\varphi }\) into \(\dd{x}\) and \(\dd{y}\), the sine of \(\theta \) cancels.

Show that \(s^{t}\) and \(s^{z }\) are constant. For \(s^{z}\): \(\vec{J} \cdot \vec{s} = 0\) is constant. For \(s^{t}\) compute.

Write \(u^{\nu } \nabla_{\nu } s^{\mu }\). 
In the end 
%
\begin{align}
\Omega = \frac{2GJ}{z^3}
\,.
\end{align}

\paragraph{Kerr} 

Horizon \(g_{rr}\) diverges, ergoregion \(g_{00} >0\). 
Hozizon is a null surface, null vector is \((1, 0,0,\Omega_{H})\), where \(\Omega_{H} = a / 2GMr\). 

The effective potential is:
\begin{align}
V _{\text{eff}} = - \frac{GM}{r} + \frac{l^2- a^2 (e^2-1)}{2 r^2} 
+ \frac{GM (l - ae)^2}{r^3}
\,.
\end{align}

The ISCO goes from \(9GM\) for extreme counterrotating to \(GM\) for extreme corotating; it is found by setting
%
\begin{align}
V _{\text{eff}} = \frac{e^2-1}{2}
\qquad \text{and} \qquad
\dv{V _{\text{eff}}}{r} = 0
\qquad \text{and} \qquad
\dv[2]{V _{\text{eff}}}{r} = 0
\,.
\end{align}

\paragraph{Penrose process}

Project \(p^{\mu } _{\text{in}} = p^{\mu } _{\text{out}} + p^{\mu } _{\text{BH}} \) onto \(-\xi^{\mu } = (-1, \vec{0})\): \(e_{BH} <0\). Also, \(l _{\text{BH}} <0\) (project onto rotating obs \(u^{\mu }\)) so angular momentum is removed from the BH.

\paragraph{Cosmology}

EFE applied to \(-\dd{t^2} + a^2 \dd{\vec{x}^2}\) with \(T^{\mu \nu }= \rho u^{\mu } u^{\nu } + p h^{\mu \nu }\) give 
%
\begin{align}
\Gamma^{0}_{ij} = \delta_{ij} a \dot{a}
\qquad \text{and} \qquad
\Gamma^{i}_{0j} = \delta_{ij} \frac{\dot{a}}{a}
\,.
\end{align}

\(R_{0i} =0\), which means \(T^{0i} = 0\). The 00 and \(ij\) equations are 
%
\begin{align}
3 H^2 = \frac{\rho}{M_P^2}
\qquad \text{and} \qquad
-2 \frac{\ddot{a}}{a}- H^2 = \frac{P}{M_P^2} 
\,,
\end{align}
%
where \(H = \dot{a} / a \),
and the conservation of the SEMT (0th component, the others vanish) is 
%
\begin{align}
\frac{\dot{\rho}}{\rho } + 3 (1+w) \frac{\dot{a}}{a} =0
\,.
\end{align}
%
Assume \(P = w \rho \) and solve EFE \(00 \) with the conservation of the SEMT. We get either \(a \sim t^{2/(3+3w)}\) or \(a \sim \exp(t)\) when \(w=-1\).
That is vacuum energy, add \(\Lambda g_{\mu \nu }\), evaluate the 00 and \(ij\) components and move it to the RHS into the \(T_{\mu \nu }\).

\paragraph{Cosmological redshift}

Start from 
%
\begin{align}
\int_A^{B} \dd{r} = \int_{t_A}^{t_B} \frac{ \dd{t}}{a}
= \int_{t_A}^{t_B} \frac{ \dd{t}}{a}
\,,
\end{align}
get 
%
\begin{align}
a_A f_A = a_B f_B
\,.
\end{align}
%


\paragraph{Hubble law}

Start from \(d_{L} = a(t) \int \dd{r} = t_0 - t_E\), then expand \(a\) to first order to find 
%
\begin{align}
d_L = \frac{z}{H_0 }
\,.
\end{align}

\paragraph{Gravitational waves}

Linearized EFE for \(g = \eta + h\) give \(\square h_{\mu 
\nu } = 0\) under the gauge choice \(g^{\mu \nu } \Gamma^
{\rho }_{\mu \nu } =0\).

The gauge transformations are derived from shifts of the position vector, the transformation is 
%
\begin{align}
h_{\mu \nu } \rightarrow h_{\mu \nu } + \partial_{(\mu } \epsilon_{\nu )}
\,,
\end{align}
%
\(\epsilon_{\mu } \) being the infinitesimal generator of the transformation. We can fix the residual gauge with vectors such that \(\square \epsilon_{\mu } =0\). 

Moving to plane wave solutions, this is the TT gauge: transverse (\(\partial_{i} h_{ij} = 0\)) and traceless (\(h^{\mu }_{\mu } = 0\)). 

\paragraph{Transformations of the basis tensors}

Slightly improper, but transform the metric, therefore the perturbation, therefore the basis tensors. 

\end{document}
