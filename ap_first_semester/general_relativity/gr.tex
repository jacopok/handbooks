\documentclass[main.tex]{subfiles}
\begin{document}

\paragraph{Length contraction and time dilation}

\begin{align}
\dd{\tau } = \frac{ \dd{t}}{\gamma } 
\qquad \text{and} \qquad
\dd{s} = \gamma \dd{x}
\,.
\end{align}



\paragraph{Energy momentum tensor}

\begin{align}
\Delta p^{\alpha } = T^{\alpha \beta } n_{\beta } \Delta V
\,,
\end{align}
%
for non interacting dust 
%
\begin{align}
T^{\alpha \beta } = n_{*} m u^{\alpha } u^{\beta }
\,.
\end{align}


\paragraph{Variational principle}

The proper time is:
\(\dd{\tau }^2 = - \dd{s^2} = -\eta_{\mu \nu } \dd{x^{\mu}} \dd{x^{\nu }} \), so the variational principle is 
%
\begin{align}
\delta \qty(\int_{A}^{B} \dd{\tau }) = 0 
\iff 
\dv{ u^{\mu}}{\tau} = 0
\,.
\end{align}

In general 
%
\begin{align}
u^{\mu } \nabla_{\mu } u^{\nu } = \dv[2]{x^{\nu }}{\tau } 
+ \Gamma^{\mu }_{\alpha \beta } u^{\alpha} u^{\beta }
\iff 
\delta \qty(\int_{A}^{B} \dd{\tau }) = 0
\,,
\end{align}
%
where \(\dd{\tau } = - g_{\mu \nu } \dd{x^{\mu }} \dd{x^{\nu }}\).

Lagrangian view: 
%
\begin{align}
\mathscr{L} = \dv{\tau }{\sigma }= \sqrt{- g_{\mu \nu } \dv{x^{\mu }}{\sigma } \dv{x^{\nu }}{\sigma }}
\,.
\end{align}

\paragraph{Light}

The 4-velocity is \emph{defined} as 
%
\begin{align}
k^{\mu } = (\omega , \vec{k})
\,,
\end{align}
%
with \(\abs{\vec{k}} = \omega \). 
Doppler effect: the observed frequency is \(-u^{\mu } k_{\mu }\), so observers moving with different velocities see different wavevectors.

\paragraph{Observers}

The measured energy of a particle with \(p^{\mu }\) by an obs with velocity \(u^{\mu }\) is \(E = - u^{\mu } p_{\mu }\). 

\paragraph{Gravitational time dilation}

\begin{align}
\frac{\Delta \tau _A}{ 1 + \Phi_{A}}
=\frac{\Delta \tau _B}{ 1 + \Phi_{B}}
\,,
\end{align}
%
and in the inertial frame \(\Delta t = \Delta \tau \) for both.

\paragraph{Christoffel}

\begin{align}
\Gamma^{\mu }_{\nu \rho } = \frac{1}{2} g^{\mu \alpha }
\qty(g_{\alpha \nu , \rho } + g_{\alpha \rho , \nu } - g_{\nu \rho , \alpha })
\,.
\end{align}

Defined by assuming \(\nabla_{\mu } A_{\nu }\) is a tensor, and the tensor differentiation law: 
%
\begin{align}
\nabla_{\mu } A_{\nu } = 
\partial_{\mu } A_{\nu } 
- \Gamma^{\rho }_{\mu \nu } A_{\rho }
\qquad \text{and} \qquad
\nabla_{\mu } A^{\nu } =
\partial_{\mu } A^{ \nu } 
+ \Gamma^{\nu }_{\mu \rho } A^{\rho }
\,.
\end{align}

\paragraph{Einstein Field Equations}

%
\begin{align}
R_{\mu \nu } - \frac{1}{2} g_{\mu \nu } R = \frac{T_{\mu \nu } }{M_P^2}
\,,
\end{align}
%
where 
%
\begin{align}
R_{\nu \rho \sigma }^{\mu } = -2 \qty(\Gamma^{\mu }_{\nu [\rho , \sigma ]}  + \Gamma^{\beta }_{\nu [\rho } \Gamma^{\mu }_{\sigma ] \beta })
\,,
\end{align}
%
\(M_P = 1 / \sqrt{8 \pi G}\) is the reduced Planck mass.



%
\begin{align}
V _{\text{eff}} = - \frac{GM}{r} + \frac{l^2}{2 r^2} \qty(1 - \frac{2GM}{r})
\,,
\end{align}
%

%
\begin{align}
V _{\text{eff}} = - \frac{GM}{r} + \frac{l^2- a^2 (e^2-1)}{2 r^2} 
+ \frac{GM (l - ae)^2}{r^3} 
\,,
\end{align}
%




\end{document}