\section{Bases}

Maxwell + Lorentz:
\[
\partial _\mu F^{\mu \nu} = j^\nu \qquad
\partial _{[\mu} F_{\nu \rho ]} = 0 \qquad 
\dv{p^\mu}{s} = e F^{\mu \nu} u_\mu
\]

Birkoff: a spherically symmetric 4-current distribution (only dependent on $(t, r)$ and with radial 3-current) with finite support ($r>r_0 \implies j^\mu = 0$) generates Coulomb-like fields for $r>r_0$. $B$ can be immediately seen to always be 0, while $E^i$ is $x^i Q / (4\pi r^3)$. 

\[
\nabla^2 \frac{1}{r} = -4\pi \delta^3 (r)
\]


\paragraph{Energy-momentum tensor}

It is the tensor $T^{\mu \nu}$ which satisfies:

\begin{equation}
p^{\mu} = \int T^{0\mu} d^3 x;
\qquad
\partial_\mu T^{\mu \nu} = 0;
\qquad
\lim_{|x| \rightarrow \infty} T^{\mu \nu} (x) |x|^3 = 0
\end{equation}

The EM one is:

\[
T^{\mu \nu} = F^{\mu \rho} F_\rho\, ^\nu + \frac{1}{4} \eta ^{\mu \nu} F^{\alpha \beta} F_{\alpha \beta} 
\]
Its $0i$ components are the Poynting vector, its $00$ component is $(E^2 + B^2)/2$.

\paragraph{Lagrange}

\begin{equation}
\pdv{\mathcal{L}}{\varphi_\mu}  
-\partial_\nu \pdv{\mathcal{L}}{\partial_\nu \varphi_\mu}
= 0
\end{equation}

For a particle in an EM field:

\begin{equation}
    I_{tot} [A_\mu (x), x^\mu (\lambda)] =
    I_{EM} [A_\mu] + I_{PART} [x^\mu] + I_{INT} [A_\mu , x^\mu]
\end{equation}

with:

\begin{align}
    I_{EM} [A_\mu] &= \int \mathcal{L} \mathrm{d}^4x; \qquad \mathcal{L} = \frac{1}{4} F^{\mu \nu} F_{\mu \nu} \\
    I_{PART} [x^\mu] &= -m \int \mathrm{d}s \\
    I_{INT} [A_\mu , x^\mu] &= \int \mathcal{L} \mathrm{d}^4x; \qquad \mathcal{L} =
    - A_\mu j^\mu
\end{align}

Where

\[
j^\mu = e \int \dd{\lambda} \dv{x^\mu}{\lambda} \delta^4 (x - x(\lambda)) 
\]

therefore 

\[
I_{INT} = \int _\gamma A_\mu \dd{x^\mu}
\]

Varying wrt the potential gives Maxwell's equations, varying wrt the position gives the Lorentz equation.

\section{Nöther}

We consider transformation for both the fields $\phi_r$ and the position $x^\mu$.
The Lagrangian might be a function of these, plus the fields' 4-derivatives.
The field index $r$ is a multi-index, and it is summed over if it is repeated!

Our transformation, which will be a symmetry, is in the form:

\begin{equation}
    \begin{cases}
    x'^\mu = x^\mu + \delta x^\mu \\
    \phi_r ' (x')  = \phi_r (x) + \Bar{\delta} \phi_r (x) \\
    \phi_r ' (x) = \phi_r (x) + \delta \phi_r (x)
    \end{cases}
\end{equation}

The two $\delta$s differ, and $\delta \partial_\mu = \partial_\mu \delta$, while $\Bar{\delta} \partial_\mu \neq \partial_\mu \Bar{\delta}$. Delta bar is the formal variation, delta is the total variation.

To the first order, the following holds:

\[
\Bar{\delta} \phi_r = \delta \phi_r + \partial_\mu \phi_r \delta x^\mu
\]

where everything is evaluated in the same untransformed point $x$.

It can be proven that the following holds:

\begin{equation}
    \partial_\mu \left(
    \delta x^\mu \mathcal{L} (\phi_r, \partial_\mu \phi_r, x) + \pdv{\mathcal{L}}{\partial_\mu \phi_r} \delta \phi_r 
    \right) = 0
\end{equation}

where the conserved expression is the Noetherian conserved current.

This $J^\mu$ can be rewritten as:

\[
J^\mu = -\delta x^\nu \Tilde{T}_\nu ^\mu + \pdv{\mathcal{L}}{\partial_\mu \phi_r} \Bar{\delta} \phi_r 
\]

with

\[
\Tilde{T} ^{\mu \nu} = \pdv{\mathcal{L}}{\partial_\mu \phi_r} \partial^\nu \phi_r - \eta^{\mu \nu} \mathcal{L} 
\]
